% !TeX root = RJwrapper.tex
\title{autoplotly - Automatic Generation of Interactive Visualizations for
Popular Statistical Results}
\author{by Yuan Tang}

\maketitle

\abstract{%
autoplotly is an R package that provides functionalities to
automatically generate interactive visualizations for many popular
statistical results supported by ggfortify package with plotly.js and
ggplot2 style. The generated visualizations can also be easily extended
using ggplot2 syntax while staying interactive.
}

\subsection{Introduction}\label{introduction}

Introductory section which may include references in parentheses
\citep{R}, or cite a reference such as \citet{R} in the text.

\subsection{Section title in sentence
case}\label{section-title-in-sentence-case}

This section may contain a figure such as Figure \ref{figure:rlogo}.

\begin{figure}[htbp]
  \centering
  \includegraphics{Rlogo}
  \caption{The logo of R.}
  \label{figure:rlogo}
\end{figure}

\subsection{Another section}\label{another-section}

There will likely be several sections, perhaps including code snippets,
such as:

\begin{Schunk}
\begin{Sinput}
library(autoplotly)
autoplotly(prcomp(iris[c(1, 2, 3, 4)]), data = iris, frame = TRUE, colour = 'Species')
\end{Sinput}
\end{Schunk}

\subsection{Summary}\label{summary}

This file is only a basic article template. For full details of
\emph{The R Journal} style and information on how to prepare your
article for submission, see the
\href{https://journal.r-project.org/share/author-guide.pdf}{Instructions
for Authors}. \bibliography{RJreferences}

\address{%
Yuan Tang\\
H2O.ai\\
2309 Wake Robin Drive\\ West Lafayette, IN 47906\\
}
\href{mailto:terrytangyuan@gmail.com}{\nolinkurl{terrytangyuan@gmail.com}}

